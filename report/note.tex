\documentclass[UTF8]{ctexart}
\usepackage{cite}
\bibliographystyle{IEEEtran}

\begin{document}
	
	\title{关于代码克隆检测的课程报告}
	\author{汤沁予}
	\date{March 11, 2020}
	\maketitle
	
\begin{abstract}
	摘要部分对全文内容做一个大体的介绍,每个段落用大约一句话简单概括。
\end{abstract}

\section{背景}
简要谈一谈什么是代码克隆,会在实际软件的维护和演化中产生什么样的影响,为什么需要进行代码克隆检测。

\section{问题定义}
介绍什么样的代码段对会被认为是克隆代码,代码克隆公认的四个类型以及几种不同的分类方法。

\section{方法和技术}
介绍代码克隆检测方法的几个大致的分类,并在每个分类中举例经典的算法进行较为详细的讨论。

\section{代码克隆检测质量分析}
简单介绍判断代码克隆检测方法质量高低的分析方法。

\section{未来趋势}
介绍代码克隆检测目前还存在的一些有待挖掘的研究方向和未来可能的发展趋势。

\bibliography{paper}
\end{document}